Amadeus hat Millionen von Problemeinträgen in einer globalen Datenbank gespeichert, zu viele um von einer Person oder selbst einem Team durchsucht zu werden. Die aktuell verfügbaren Werkzeuge im diese Datenbank zu durchsuchen sind nicht ausreichend um schnell und einfach zu suchen. Aus diesem Grund wird es immer schwerer an die wertvollen Informationen in der Datenbank zu kommen, weil einfach zu viel Information vorhanden ist. Das ist die Hauptursache für viele duplizierte Einträge und Einträge die falschen Teams zugewiesen wurden: Die Teams wissen nicht genau was andere tun. Das führt dazu, dass das Zeit, um ein Problem zu lösen, steigt und damit einhergehend die Kundenzufriendenheit sinkt.

Die folgende Arbeit zeigt, wie die vorgeschlagene Lösung namens ``{\reporttitle}'' versucht das Problem, als Twitter-ähnliche Fassade vor der Problemdatenbank, zu lösen und dabei Entwicklern und Produkt-Managern zu erlauben die Datenbank mit einfachen, kurzen Nachrichten zu durchsuchen.