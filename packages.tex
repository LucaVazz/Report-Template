%!TEX root = main.tex
%!TEX program = xelatex
%!BIB program = biber

%% GENERAL PACKAGES

%%   specific Packages only needed for the Layout are loaded inside layout.tex


% general typesetting:
\usepackage{polyglossia}


% some fine helpers:
\usepackage{amsmath,amssymb,amsfonts}

\usepackage{lipsum}

\usepackage{gensymb}%ensures the display of some unit-symbols
                    %   provides \degree, \celsius, \perthousand, \micro and \ohm

\usepackage{calc}%provides arithmetics on \set­counter, \ad­dto­counter, \setlength, and \ad­dtolength

\usepackage{enumerate}


% generally used layout:
\usepackage{scrlayer-scrpage}

\usepackage{setspace}%set spacing after paragraphs
                     %   providees \doublespacing, \onehalfspacing, \singlespacing and
                     %   \setstretch{baselinestretch}

\usepackage{tabto}%add tabs to a specific length

\usepackage{changepage}

\usepackage{float}%eases work with floating objects (e.g. figures and tables)
\usepackage{caption}%customize captions for floating environments


% generally used eye-candy styling:
\usepackage{marvosym}
\usepackage{adforn}

\usepackage{color}
\usepackage[usenames,dvipsnames,table]{xcolor}

\usepackage{xhfill}


% tables:
\usepackage{longtable,tabu}
\usepackage{booktabs}

\usepackage[newcommand]{ragged2e}


% citations and references:
\usepackage[autostyle]{csquotes}

\usepackage[babel=true]{microtype}

\usepackage[backend=biber, style=authoryear]{biblatex}
\addbibresource{resources/references.bib}
\usepackage[footnote]{acronym}

\usepackage[hidelinks]{hyperref}


% graphics:
\usepackage{graphicx}
\usepackage{wrapfig}

\usepackage{tikz}
\usetikzlibrary{matrix}

\graphicspath{{./resources/images/}}

\usepackage{pdfpages}%include external PDF pages

\usepackage{pgfplots}%utilities to draw 2D and 3D plots
\pgfplotsset{compat=newest}


% listings
\usepackage{listings} 

\lstdefinelanguage{HTML5}{
    sensitive=true,
    keywords={%
    % JavaScript
    typeof, new, true, false, catch, function, return, null, catch, switch, var, if, in, while, do, else, case, break,
    % HTML
    html, title, meta, style, head, body, script, canvas,
    % CSS
    border:, transform:, -moz-transform:, transition-duration:, transition-property:,
    transition-timing-function:
    },
    % http://texblog.org/tag/otherkeywords/
    otherkeywords={<, >, \/},   
    ndkeywords={class, export, boolean, throw, implements, import, this},   
    comment=[l]{//},
    % morecomment=[s][keywordstyle]{<}{>},  
    morecomment=[s]{/*}{*/},
    morecomment=[s]{<!}{>},
    morestring=[b]',
    morestring=[b]",    
    alsoletter={-},
    alsodigit={:}
}

\definecolor{lightgray}{rgb}{.9,.9,.9}
\definecolor{darkgray}{rgb}{.4,.4,.4}
\definecolor{purple}{rgb}{0.65, 0.12, 0.82}
\lstdefinelanguage{JavaScript}{
  keywords={break, case, catch, continue, debugger, default, delete, do, else, false, finally, for, function, if, in, instanceof, new, null, return, switch, this, throw, true, try, typeof, var, void, while, with},
  morecomment=[l]{//},
  morecomment=[s]{/*}{*/},
  morestring=[b]',
  morestring=[b]",
  ndkeywords={class, export, boolean, throw, implements, import, this},
  keywordstyle=\color{blue}\bfseries,
  ndkeywordstyle=\color{darkgray}\bfseries,
  identifierstyle=\color{black},
  commentstyle=\color{purple}\ttfamily,
  stringstyle=\color{red}\ttfamily,
  sensitive=true
}

\lstset{
   language=JavaScript,
   backgroundcolor=\color{lightgray},
   extendedchars=true,
   basicstyle=\footnotesize\ttfamily,
   showstringspaces=false,
   showspaces=false,
   numbers=left,
   numberstyle=\footnotesize,
   numbersep=9pt,
   tabsize=2,
   breaklines=true,
   showtabs=false,
   captionpos=b
}


% tables:
\usepackage{colortbl}%use colors in tables
\usepackage{booktabs}%better horizontal lines in tables 
                     %   provides \toprule[<breite>], \midrule[<breite>], 
                     %   \cmidrule[<breite>](trim){a-b} and \bottomrule 
\usepackage{multirow}%merge several cells over rows or columns
                     %   provides \multirow{Zeilen}{Breite}{Inhalt} and  
                     %   \multicolumn{Spalten}{Ausrichtung}{Inhalt}
\usepackage{supertabular,longtable}%makes tables stretching over multiple pages possible



% Keep this as the very last item!
\usepackage{scrhack}%smooths edges at playing along with KOMA-Script
