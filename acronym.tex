%% Hier alle Abkürzungen eintragen und mit \ac{Abk.} benutzen.
%% ac{ABK} setzt die Abkürzung automatisch 
%% acl{Abk} schreibt die lange Version
%% wenn an einen /ac Befehl ein p angehängt wird /acp wird die Pluralform
% verwendet
 
\acro{RND}{Research \& Development}
\acro{PTR}{Problem Tracking Record}
\acro{IR}{Investigation Record}
\acro{WO}{Work Order}
\acro{CR}{Change Request}
\acro{TF}{Term Frequency}
\acro{IDF}{Inverse Document Frequency}
\acro{BM25}{Best Match 25}
\acro{HTML}{Hypertext Markup Language}
\acro{IO}{Input / Output}
\acro{NPM}{A package manager for NodeJS}
\acro{MVC}{Model View Controller}
\acro{JSON}{JavaScript Object Notation}
\acro{BSON}{Binary JSON}
\acro{SQL}{Structured Query Language}
\acro{CRUD}{Create Read Update Delete}
\acro{FFT}{Final-Form Text}
\acro{HTTP}{Hypertext Transfer Protocol}
\acro{API}{Application Programming Interface}
\acro{UI}{User Interface}
\acro{REST}{Representational State Transfer}